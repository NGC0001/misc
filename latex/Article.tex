% LaTex Article template

\message{Messages here will be put out to screen when being compiled.}
\listfiles
%------------------------------------------------------------------------
\documentclass[12pt, a4paper, onecolumn, notitlepage]{article}
%------------------------------------------------------------------------
\usepackage[%CJKnumber = true,%是否使用CJKnumber宏包
            xeCJKactive = true
            ]{xeCJK}
\xeCJKsetup{CJKmath = true,%直接在数学环境中使用中文
            %CheckSingle = true,%避免单个汉字占据一段的最后一行
            AutoFallBack = true,%自动使用后备字体输出生僻字
            AutoFakeBold = {true},%自动使用伪粗体 AutoFakeBold = {<true|数字>}
            AutoFakeSlant = {true},%自动使用伪斜体 AutoFakeSlant = {<true|数字>}
            %xeCJKspace = true,%保留中文文字间的空格
            %AllowBreakBetweenPuncts = true,%允许在CJK左标点和CJK右标点间断行
            xeCJKactive = true
            }

\input{ChineseEnv}

\XeTeXlinebreaklocale "zh"
\XeTeXlinebreakskip = 0pt plus 1pt minus 0.1pt
\defaultfontfeatures{Mapping=tex-text}%如果没有它,一些tex特殊字符无法正常使用,比如连字符
\setlength{\parindent}{2.3em}%设定段首缩进
%------------------------------------------------------------------------
\usepackage{verbatim}
\usepackage{makeidx}
%\usepackage{showidx}
\renewcommand\indexname{索~引}
\makeindex%激活索引命令
\usepackage{color}
\usepackage{makecell}
\usepackage{syntonly}
%\syntaxonly%只作语法检查,并不编译
\usepackage{CJKfntef}%实现汉字加点等
\usepackage{amsmath,amssymb}
\usepackage{bm}
%------------------------------------------------------------------------
\usepackage{graphicx}
%在文档中这样使用:\includegraphics[key=value,...]{file}
%几个主要关键词:
%width   把图形缩放到指定的宽度,可以是绝对值,也可以是相对值,如\textwidth
%height  把图形缩放到指定的高度,可以是绝对值,也可以是相对值,如\textheight
%angle   逆时针旋转图形
%scale   缩放图形
%------------------------------------------------------------------------
% \newtheorem{name}[counter]{text}[section]
% name是标记,text是出现在打印文档中的名称
% counter表明该'定理'是否与某个'定理'共用计数器。section表明计数是否与文档某个段落层次关联
% 定义之后,在文档中如下使用:
% \begin{name}[别名]
% 内容
% \end{name}
\newtheorem{MyTheorem}{杨氏定理}
%------------------------------------------------------------------------
\usepackage{hyperref}
\hypersetup{pdfauthor={杨云鹏},
            pdftitle={},
            pdfsubject={},
            pdfkeywords={},
            %pdfpagemode={FullScreen},
            colorlinks={true},
            linkcolor={red}
            }
%------------------------------------------------------------------------
%\includeonly{filenamelist}%在文档中所有的\include命令中,只把本命令所指定的文件包含进来
\pagestyle{headings}
%\linespread{factor}%定义行距,\linespread{1.3} 将产生 1.5 倍行距,而 1.6 则产生双倍行距。缺省情况下的行距为 1。
%\setlength{\parindent}{0pt}%设定首行缩进为0
%\setlength{\parskip}{1ex plus 0.5ex minus 0.2ex}%设定段间距为 0.8ex 到 1.5ex
%\setlength{parameter}{length}%设定页面参数
%\addtolength{parameter}{length}%改变页面参数
%------------------------------------------------------------------------
\begin{document}
%------------------------------------------------------------------------
%下面的命令对book类型的文档有效
%\frontmatter%把页码变为罗马数字
%\mainmatter%打开阿拉伯页码计数器,对页码重新计数
%\backmatter
%------------------------------------------------------------------------
\title{\textbf{\textcolor[rgb]{0.00,1.00,1.00}{\LaTeXe 模板}}}
\author{\textbf{\textcolor[rgb]{1.00,0.00,1.00}{杨云鹏\thanks{感谢我的信仰!}}}\and
        \textbf{\textcolor[rgb]{1.00,1.00,0.00}{小伙伴\thanks{感谢我的小伙伴!}}}
        }
\date{\textcolor[rgb]{0.00,1.00,0.00}{\today}}
\maketitle
%------------------------------------------------------------------------
\renewcommand{\contentsname}{目 录}
\thispagestyle{headings}
\tableofcontents%[hideallsubsections]
\listoffigures
\listoftables
%------------------------------------------------------------------------
%\include{tex file}%在处理插入的文件之前会开始新的一页。
%\thispagestyle{plain/headings/empty}
%\sloppy%降低排版标准
%\fussy%恢复排版默认标准
%\indent%缩进本来没有缩进的段落,只有 \parindent 不为零时才有效果。为了缩进章节标题之后的第一个段落,可以使用indentfirst包。
%\noindent%创建一个不缩进的段落。
%------------------------------------------------------------------------
\section{空白和空行}
多个空白字符(空格和跳格)视同一个,作用是空格;多个空行视同一个,作用是结束段落。 一段中的分行是由\TeX 自动进行的。

\verb+\\+和\verb-\newline-用来强制换行,但并不结束段落。在强行段行后,还禁止分页。
\verb-\par-也是换行命令。

$\backslash${}newpage用来分页。

\verb+\linebreak[n], \nolinebreak[n], \pagebreak[n] +和\\
\verb+\nopagebreak[n]+用于向LaTeX提出(不)分行/(不)分页的建议。

单个换行符视为空格;行首空格被忽略。
\section{命令}
保留字符:\# \$ \% \^{} \& \_ \{ \} \~{} $\backslash${}

命令:1、反斜线加只包含字母的命令名,命令后的空格、数字或其他非字母字符标志命令的结束;2、反斜线和一个特殊字符。

\TeX{} is fine! Ha\ldots{} Today is \today.
\section{注释}
\%用来注释,或用来分割不允许有空格或换行的输入文本。

这里有一些你看不到的%
\begin{comment}
这里用了verbatim宏包提供的comment环境。
注意,这方法不能使用在诸如数学环境等一些复杂环境中。
\end{comment}
注释文本。
\section{排版和一些符号}
字体从小到大:\\
\verb-\tiny \scriptsize \footnotesize \small-\\
\verb-\normalsize \large \Large \LARGE \huge \Huge-\\

除了命令,字体大小也可用环境的方式来实现。如Large环境。

大括号(curly braces)扮演了一个重要角色。它们建立所谓的组。组限制了大多数命令的作用范畴。

使用$\backslash$hyphenation\{word list\}和$\backslash$-来指导\LaTeX{}断字。

\mbox{These words are forced to be placed in one line.}

\fbox{In one line and surrounded by a box as well.}

\emph{强调} \underline{下划线}
``双引号'' `单引号'

四种短划标点: 连字号- 短破折号-- 长破折号--- 减号$-$

波浪号: \~{} $\sim$

$-30\,^{\circ}\mathrm{C}$%mathrm只对于较短的项才起作用,空格不起作用,重音字符也不起作用,还有mathcal, mathit等等
\quad \ldots \quad $\cdot$

连字: 不是shelfful而是shelf\mbox{}ful

空格前的反斜线产生一个不能伸长的空\ 格

\~{}产生一个不能伸长的空~格,且禁止断行

句号前的命令$\backslash$@说明此句号是句末,
即使它紧跟一个大写字母(一般句号紧跟大写字母表示缩写)

$\backslash$frenchspacing禁止在句号后插入额外空间
(这样$\backslash$@就不必要了)\\ \text{}

水平距离:这是\hspace{1.5cm}一段1.5厘米的空白。\\ %如果这个水平距离在行首或者行末应该消失的话,用命令\hspace*代替\hspace。
x\hspace{\stretch{1}}x\hspace{\stretch{3}}x.\\ %\stretch{n}产生一个将一行的宽度充满的橡皮长度。
                                             %如果两个\hspace{\stretch{n}}命令位于同一行,那么它们将根据伸缩因子分配空间。

垂直距离:\\
\verb+\vspace{length}+通常用于两个空行之间。
使用命令\verb-\vspace*-避免这个额外的行距存在于页的顶部和末尾。
命令 \verb-\vspace \stretch 和 \pagebreak- 结合使用可以在页的最后一行输出文本,
也可以用来保证文本在页面上垂直居中。\\ \bigskip

使用命令 \verb-\bigskip- 和 \verb-\smallskip- 可以获得一个预定义的垂直距离。
\begin{comment}
\settoheight{lscommand}{text}
\settowidth{lscommand}{text}
\settodepth{lscommand}{text}
以上三个命令允许获得一个文本串的宽度、高度以及深度。如:\settowidth{\parindent}{This Paragraph:}
\end{comment}
\section{注音符号和特殊符号}
\^o \"a \"i \"\i{} \'e \`e \~n \=e \.e \u e \v e \H e \aa{} \AA{}

\c c \b e \d e

!` ?` \ss{} \oe{} \OE{} \ae{} \AE{} \o{} \O{} \l{} \L{} \i{} \j{}
\section{标题,章和节}
\label{jie:biaoti}
对于Article,有 section, subsection, subsubsection, paragraph, subparagraph 等分节命令。

对于book和report,还有part, chapter两个命令。
\section*{此节不出现于目录,也不被编号}
参看第\pageref{jie:biaoti}页第\ref{jie:biaoti}节。
\section[短标题的使用]{短标题:这里是文档中的标题}
参看第\pageref{jie:biaoti}页第\ref{jie:biaoti}节\footnote{这是一个脚注。}。
\subsection{居中}
flushleft, flushright, center用于控制对齐方式。
可以用\verb+\flushleft+ \verb|\flushright|, \verb-\center-等命令。

也可以用 \verb&\begin{} \end{}& 环境方式。\footnote{这里使用了逐字打印命令,
其中的+, |, -等是分隔符,但*和空格不能用作分隔符。}
\section{列举}
\flushleft
\begin{enumerate}
 \item You can mix the list environments to your taste:
       \begin{itemize}
        \item But it might start to look silly.
        \item[-] With a dash.
       \end{itemize}
 \item Therefore remember:
       \begin{description}
        \item[Stupid] things will not become smart because they are in a list.
        \item[Smart] things, though, can be presented beautifully in a list.
       \end{description}
\end{enumerate}
\section{引用}
quote, quotation, verse是引用环境。
quotation用于几个段落的长引用,verse用于诗歌的引用。
\begin{verse}
Humpty Dumpty sat on a wall:\\
Humpty Dumpty had a great fall.\\
All the King’s horses and all
the King’s men\\
Couldn’t put Humpty together again.
\end{verse}
\section{逐字打印}
\verb+\begin{verbatim}+和\verb+\end{verbatim}+用于逐字环境。

此外,有\verb+\verb+和\verb+\verb*+命令,还有verbatim*环境。
带星号表示标记空格。

例如:\verb*+空 格+

\verb-\verbatiminput{filename}-命令可以把一个 ASCII 码的文本文件包含到文档中来,
就好像它是在verbatim环境中一样。
\section{表格}
\begin{tabular}{|l|c|r|p{2cm}|}
\hline%表格的水平线
左对齐的列 & 居中的列 & 右对齐的列 & 宽度为2cm,自动换行的列 \\
\cline{1-1} \cline{4-4} \verb+\cline{i-j}+ & 添加部分表线 & i, j表示起始和终止列号 & \\
\hline
\hline
\verb+\hline+ & & 产生整体表线 &
\end{tabular}
\\
\begin{tabular}{@{} l @{}}%@{}用于设定分隔符
\hline
no leading space\\
\hline
\end{tabular}
\\
\begin{tabular}{l}
\hline
leading space left and right\\
\hline
\end{tabular}
\\ \text{} \\
\begin{tabular}{c r @{.} l}
Pi Expression & \multicolumn{2}{|c}{Value}\\
%multicolumn用于跨列内容,它两边的分割符要重设
\hline
$\pi$ & 3&1416 \\
$\pi^{\pi}$ & 36&46 \\
$(\pi^{\pi})^{\pi}$ & 80662&7 \\
\end{tabular}

%\nomakegapedcells
\begin{tabular}{|l|c|c|}\hline
\diaghead(-4,1){\hskip4.2cm}{Diag \\Column Head I}{Diag Column \\Head II} &
\thead{Second\\column} & \thead{Third\\column} \\ \hline
\end{tabular}
\section{浮动体}
即figure和table环境。这两个环境支持放置说明符。

\begin{tabular}{c l}
放置说明符 & 浮动体允许放置位置 \\
\hline
h & 浮动体就放在当前页面上。这主要用于小浮动体 \\
t & 放在页面顶部 \\
b & 放在页面底部 \\
p & 放在一专门页面,仅含一个浮动体 \\
! & 忽略阻止浮动体放置的大多数内部参数 \\
\end{tabular}\\
缺省值为[tbp]。不建议仅使用单个放置说明符,特别是单个[h]。

下面用\verb+\makebox+画一个方形插入文档。
事实上,这可以用来给图片预留空间~{\ref{white-figure}}。
\begin{figure}[!hbp]
\makebox[\textwidth]{\framebox[5cm]{\rule{0pt}{5cm}}}
\caption[浮动体]{ $5\times 5\ cm^2$的预留空间。} \label{white-figure}
\end{figure}
用 \verb+\clearpage 甚至 \cleardoublepage+ 命令\LaTeX 立即安排
等待序列中所有剩下的浮动体,并且开一新页。
命令 \verb-\cleardoublepage- 会新开奇数页面。
\section{保护脆弱命令\protect\footnote{这就是一个例子。}}
\verb+\footnote 或 \phantom +是脆弱命令的例子。
\verb-\protect -仅仅保护紧跟其右侧的命令,连它的参量也不惠及,
在大多数情形下,过多的\verb- \protect -并不碍事。
\section{数学公式}
在数学模式中:
\begin{itemize}
 \item[1.]空格和分行都将被忽略。所有的空格或是由数学表达式逻辑的衍生,或是由特殊的命令得到。
 \item[2.]不允许有空行,每个公式中只能有一个段落。
 \item[3.]每个字符都将被看作是一个变量名并以此来排版。
如果希望在公式中出现普通文本(使用正体字并可以有空格),那么必须使用命令\verb- \textrm{...} -来输入这些文本。
\end{itemize}

段落内部数学表达式用:\verb+ \( \), $ $, \begin{math} \end{math}+,
如:\(\tau\epsilon\chi\) , My $\heartsuit$ , \begin{math}a+b=c\end{math},
$\iint_{D}\,\mathrm{d}x\mathrm{d}y$ , $\sum_{k=1}^n$ , $\lim_{n \to \infty}$

\text{}

大的数学式子用\verb+ \[ \], \begin{displaymath} \end{displaymath}+\\
\[\iint_{D}\,\mathrm{d}x\mathrm{d}y\int_0^1\,\sum_k\vec {S_k}\vec S_k\,\mathrm{d}z\stackrel{!}{=}\sum_{i=1}^n\vec {P_i}{\vec P}_i\]
%\stackrel将第一项中的符号以上标大小放在处于正常位置的第二项上。
\begin{displaymath}
\lim_{n \to \infty}\,=\,\frac{\pi^2}{6}\cdot\prod_{i}^{1\rightarrow \infty}\sigma_i
\end{displaymath}

\text{}

equation环境会对公式编号,用label, ref引用公式。
eqnarray可以垂直对齐,amsmath提供了一些更好的垂直环境(split, align等)。

\text{}

$\sqrt[100]{\sqrt{100}*\surd[100]}$ \\[10pt]%设置行间距
$\overline{\underline{m}+\underline{n}}$ \\
$\widetilde{xyz} \quad \widehat{abc} \quad \vec a \qquad \overleftarrow{AB} \qquad \overrightarrow{AB} \quad x'y''$ \quad x'y'' \\
$a\bmod b$ \qquad \qquad $a\pmod b$ \\
$\underbrace{ a+b+\cdots+z }_{26}$ \quad $\overbrace{ a+b+\cdots+z }^{26}$ \\
$\ldots\,\cdots\,\vdots\,\ddots$ \\

\[{n \choose m} \qquad {l \atop k+1}\]
\begin{displaymath}
1+\left(\frac{1}{\pi^2}\right)\neq\left\{\frac{1}{\pi}\right.\qquad\big\|\Big\|\bigg\|\Bigg\|
\end{displaymath}  
\begin{displaymath}
1+\big(\frac{1}{\pi^2}\Big)\neq\bigg(\frac{1}{\pi}\Bigg)
\end{displaymath}  
\\[30pt]

函数名通常用罗马字体正体排版,而不是像变量名一样用意大利体排版。
\begin{verbatim}
\arccos \cos   \csc  \exp  \ker    \limsup \min
\arcsin \cosh  \deg  \gcd  \lg     \ln     \Pr
\arctan \cot   \det  \hom  \lim    \log    \sec
\arg    \coth  \dim  \inf  \liminf \max    \sin
\sinh   \sup   \tan  \tanh 
\end{verbatim}
\subsection{数学空格}
\begin{tabular}{l|l|l}
\verb-\,-     &  $|\,|$     & 3/18 quad \\
\verb-\:-     &  $|\:|$     & 4/18 quad \\
\verb-\;-     &  $|\;|$     & 5/18 quad \\
\verb*-\ -    &  $|\ |$     & 中等大小空格 \\
\verb-\quad-  &  $|\quad|$  & 大空格,对应于目前字体中字符`M'的宽度 \\
\verb-\qquad- &  $|\qquad|$ & 大空格 \\
\verb-\!-     &  $|\!|$     & -3/18 quad \\
\end{tabular}
\subsection{垂直对齐}
\subsubsection{array环境}
array有些类似于tabular环境,使用 \verb-\\- 命令来分行。也可以在 array 环境中画线。
\begin{displaymath}
\mathbf{X} =
\left( \begin{array}{ccc}
x_{11} & x_{12} & \ldots \\
x_{21} & x_{22} & \ldots \\
\vdots & \vdots & \ddots
\end{array} \right)
\end{displaymath}
\begin{displaymath}
\left(\begin{array}{c|c}
1 & 2 \\
\hline
3 & 4
\end{array}\right)
\end{displaymath}
\subsubsection{eqnarray 和 eqnarray* 环境}
eqnarray 和 eqnarray* 环境类似于 \{rcl\} 形式的三列表格。
在 eqnarray 中,每一行都会有一个方程编号。eqnarray* 不对方程进行编号。
\setlength\arraycolsep{2pt}%由于中间列两边的空格默认很大,用这个命令调节
\begin{eqnarray}
\sin x & = & x -\frac{x^{3}}{3!}+\frac{x^{5}}{5!}-{}\nonumber\\
       &   & -\frac{x^{7}}{7!}+\cdots
\end{eqnarray}%nonumber表示该行不编号
\subsection{幻影}
\verb-\phantom- 命令可以为不在最终输出中出现的字符预留空间。
\begin{displaymath}
\Gamma_{ij}^{\phantom{ij}k}
\qquad \textrm{versus} \qquad
\Gamma_{\phantom{k}ij}^{k}
\end{displaymath}
\subsection{自定义定理}
\begin{MyTheorem}[绝对零度不可达到定理]
时运不济,命途多舛。
\end{MyTheorem}
\subsection{数学字体}
在数学模式中,字体大小用四个命令来设定:\\
\verb-\displaystyle, \textstyle, \scriptstyle, \scriptscriptstyle- \\
改变字体时,相应的定界符也要改变。如:
\begin{displaymath}
\mathop{\mathrm{corr}}(X,Y)=
\frac{\displaystyle\sum_{i=1}^n(x_i-\overline x)(y_i-\overline y)}
{\displaystyle\biggl[\sum_{i=1}^n(x_i-\overline x)^2\sum_{i=1}^n(y_i-\overline y)^2\biggr]^{1/2}}
\end{displaymath}
这个例子中,需要比标准的 \verb-\left[ \right]- 还要大一些的括号。\\ \text{} \\

字体改变命令 \verb-\mathbf- 给出粗体字母,但是这些是罗马字体(竖直的),而数学符号通常是斜体。
有一个 \verb-\boldmath- 命令,但是这只能用于数学模式之外。对于符号也是如此。
\section{参考文献}
用 thebibliography 环境产生一个参考文献。在一个参考文献中每个条目以如下命令开头:
{\center{\verb-\bibitem{marker}-}}\\
然后使用 marker 在正文中引用这本书、这篇文章或者论文。
{\center{\verb-\cite{marker}-}}\\
参考文献条目的编号是自动生成的。\\
Book~\cite{ReferenceBook} is a good book.
\renewcommand\refname{参考文献}
\begin{thebibliography}{99}%{99}告诉 LATEX 参考文献条目的编号不会比数字 99 更宽。
\bibitem{ReferenceBook} Y.~My True Love: \textit{Chinese \TeX}, \textsl{YunPeng}, Volume~9, Issue~1 (1992)
\end{thebibliography}
%意大利斜体\textit{};slanted斜体\textsl{}
\section{索引命令的使用}
\index{普通项}  \text{}
\index{普通项!Affiliated Item}  \text{}\\
\index{FormateOfItem1@\textbf{索引项粗体格式}}  \text{}\\
\index{FormateOfItem2@\textsl{索引项斜体格式}}  \text{}\\
\index{页码粗体格式|textbf}  \text{}\\
\index{页码斜体格式|textit}  \text{}\\
\section{\LaTeXe 中的盒子}
LATEX 通过盒子来建立整个文档的布局。
每个字符都是一个小的盒子,这些盒子连接起来构成单词,单词本身连接起来构成一行。
单词之间的连接是一个橡皮连接, LATEX 将自动进行调整使得单词将恰好构成一行。
几乎可以把任何可见元素(包括盒子自身)放到一个盒子中。
然后 LATEX 将会像处理单个字母一样处理这个盒子。\\

\verb-\parbox[pos]{width}{text}-\\
\verb-\begin{minipage}[pos]{width} text \end{minipage}-\\
参数 pos 可以取以 c, t 或 b,这个参数用于控制盒子里文本的垂直位置。\\[20pt]

\verb-\mbox-事实上就是一个盒子。\\
\verb-\makebox[width][pos]{text}-用于处理水平对齐的材料。
除绝对数值外,也可以传递\verb-\width- \verb-\height- \verb-\depth- \verb-\totalheight-给width。
这几个值是测量盒子内部文本来获得的(注意,width是盒子从外部看起来的宽度)。
参数 pos 接受一个字符: c–居中、l–靠左、r–靠右和 s–将文本均匀分布到整个盒子中。
命令 \verb+\framebox 和 \makebox+ 完成同样的工作,不同之处在于它在内部文本的周围画出一个矩形框。\par
\makebox[\textwidth]{c e n t r a l}\par
\makebox[\textwidth][s]{s p r e a d}\par
\framebox[1.1\width]{Guess I’m framed now!} \par
\framebox[0.8\width][r]{Bummer, I am too wide} \par
\framebox[1cm][l]{never mind, so am I}Can you read this?\\[20pt]

\verb+\raisebox{lift}[depth][height]{text}+能够定义盒子的高度。
\raisebox{0pt}[0pt][0pt]{\Large%
\textbf{Aaaa\raisebox{-0.3ex}{a}%
\raisebox{-0.7ex}{aa}%
\raisebox{-1.2ex}{r}%
\raisebox{-2.2ex}{g}%
\raisebox{-4.5ex}{h}}}
he shouted but not even the next one in line noticed that
something terrible had happened to him.
\section{标尺和支撑}
\verb-\rule[lift]{width}{height}-可以用来产生水平方向和垂直方向的线条。

一种特殊的应用就是没有宽度只有高度的标尺。
在专业的出版术语中,这被称为支撑(Struts)。
它被用来保证文档的一部分具有一个确定的高度最小值。
\section{定制\LaTeX}
定义自己的命令:\verb+\newcommand{name}[num]{definition}+\\
不可重定义已有命令。num 是可选的,用于指定命令所需的参数数目(最多9个)。用 \#1 等使用参数。\\
\verb+\renewcommand+使用与命令\verb-\newcommand-相同的语法。用于重写已有命令。\\
\verb-\providecommand- 完成与 \verb-\newcommand- 相同的工作。但如果命令已存在,它将会被忽略。\\ \text{}

定义自己的环境:\verb-\newenvironment{name}[num]{before}{after}-\\
在参数 before 中提供的内容将在被命令包含的文本之前处理,
而在参数 after 中提供的内容将恰好在 \verb-\end{name}- 的前面处理。\\ \text{}

定义自己的宏包:\\
写一个宏包的基本工作就是将原本很长的文档导言(例如包含了很多新命令和新环境)拷贝到一个分离的文件中去,
这个文件需要以 .sty 结尾。
还需要使用一个专用的命令: \verb-\ProvidesPackage{package name}-。
这个命令应该在你的包的最前面使用,用于告诉 LaTeX 宏包的名称,
从而允许 LaTeX 在你尝试两次引入同一个宏包的时候给出一个错误信息。
%------------------------------------------------------------------------
\renewcommand{\appendixname}{附录}
\appendix
\section{结束语}
\text{} \\ \text{}

\textbf{\Huge{\textcolor[rgb]{1.00,0.00,0.00}{\kai 谢谢!}}}
\textbf{\Huge{\textcolor[rgb]{0.00,0.50,1.00}{\hei Thanks!}}}
%------------------------------------------------------------------------
\printindex
%------------------------------------------------------------------------
\end{document}
