\message{Messages here will be put out to screen when being compiled.}
\listfiles
%------------------------------------------------------------------------
\documentclass{beamer}
\mode<presentation>

\usetheme{Hannover}%在这里选择beamer内建主题,它包括了颜色、字体、列举签样式、Logo位置等等等,不过可以被更具体的设置覆盖
%\logo{\includegraphics[]{/home/sagittarius/LaTeX/figure/pkulogotrans.png}}
%\usepackage{beamerthemeshadow}

%更细的主题分为color,font,inner,outer四种
%\usecolortheme{seahorse}
%\usefonttheme{serif}
%\useinnertheme{rounded}
%\useoutertheme{sidebar}

%更为详细的设置
%\setbeamercolor{itemize item}{fg=orange,bg=green}
%\setbeamercolor{structure}{fg=gray}
%\setbeamercolor{normal text}{fg=blue}
%\setbeamerfont{frametitle}{series=\bfseries}

%\setbeamertemplate{blocks}[rounded][shadow=true]
%\setbeamertemplate{navigation symbols}{}%导航工具
%\setbeamertemplate{items}[ball]%设定列举签的样式,也可以是circle,rectangle,default(三角形)等
%\setbeamertemplate{theorems}[numbered]%设置让定理也编号
%\setbeamertemplate{caption}[numbered]%设置让图表也编号
%\setbeamertemplate{background}{\includegraphics[height=\paperheight,width=\paperwidth]{%
                                %/home/sagittarius/LaTeX/figure/BackgroundPicture}}%设定背景图片

%\beamertemplateshadingbackground{red!10}{yellow!10}%背景色
%\beamertemplatesolidbackgroundcolor{green!10}%背景色
%\beamertemplategridbackground[0.5cm]%背景网格
%\beamertemplatenumberedballsectiontoc
%\beamertemplateboldpartpage

%\definecolor{blendedred}{rgb}{0.7,0.2,0.2}%定义颜色

%注:Beamer内建了 definition, lemma, theorem, corollary, proof, example, examples 等定理环境
%------------------------------------------------------------------------
\usepackage[%CJKnumber = true,%是否使用CJKnumber宏包
            xeCJKactive = true
            ]{xeCJK}
\xeCJKsetup{CJKmath = true,%直接在数学环境中使用中文
            %CheckSingle = true,%避免单个汉字占据一段的最后一行
            AutoFallBack = true,%自动使用后备字体输出生僻字
            AutoFakeBold = {true},%自动使用伪粗体
            AutoFakeSlant = {true},%自动使用伪斜体
            %xeCJKspace = true,%保留中文文字间的空格
            %AllowBreakBetweenPuncts = true,%允许在CJK左标点和CJK右标点间断行
            xeCJKactive = true
            }

\input{ChineseEnv}

\XeTeXlinebreaklocale "zh"
\XeTeXlinebreakskip = 0pt plus 1pt minus 0.1pt
\defaultfontfeatures{Mapping=tex-text}%如果没有它,一些tex特殊字符无法正常使用,比如连字符
\setlength{\parindent}{2.3em}%设定段首缩进
%------------------------------------------------------------------------
\usepackage{verbatim}
\usepackage{color}
\usepackage{syntonly}
%\syntaxonly%只作语法检查,并不编译
\usepackage{CJKfntef}%实现汉字加点等
\usepackage{amsmath,amssymb}
\usepackage{graphicx}
\usepackage{multimedia}
%------------------------------------------------------------------------
\usepackage{pgf,pgfarrows,pgfnodes,pgfautomata,pgfheaps}
\def\pgfsysdriver{pgfsys-xdvipdfmx.def}
%\usepackage{tikz}
%\pgfdeclaremask{pkulogomask}{/home/sagittarius/LaTeX/figure/pkulogo-mask.png}%声明蒙板
%------------------------------------------------------------------------
\hypersetup{%pdftitle={},
            pdfsubject={Hello, world!},
            %pdfkeywords={},
            pdfauthor={杨云鹏},
            %pdfpagemode={FullScreen},
            colorlinks={true},
            linkcolor={red}
            }
%------------------------------------------------------------------------
%\includeonly{filenamelist}%在文档中所有的\include命令中,只把本命令所指定的文件包含进来
%------------------------------------------------------------------------
\begin{document}
\setbeamercovered{invisible}%也可设置为transparent,dynamic,highly dynamic,须在文档环境内设置
%------------------------------------------------------------------------
%注意:下面这些定义提纲的命令会干扰\label命令的标签序号,因此不建议使用
%\AtBeginSection[]{\frame{\frametitle{\secname}
%                         \tableofcontents[currentsection,hideallsubsections]
%                         }
%                  }
%\AtBeginSubsection[]{\frame{\frametitle{\subsecname}
%                            \tableofcontents[current,currentsubsection]
%                            }
%                     }
%------------------------------------------------------------------------
\title{\textbf{\textcolor[rgb]{0.00,1.00,1.00}{Beamer模板}}}
\subtitle{\textcolor[rgb]{1.00,0.00,0.50}{欢迎观看}}
\subject{Hello, world!}
\institute[PHY, PKU]{\textbf{\kai 北京大学物理学院}}
\author[杨云鹏,小伙伴]{\textcolor[rgb]{1.00,0.00,0.00}{杨云鹏}\\ \texttt{yangyunpeng@pku.edu.cn}
        \\ \text{} \\
        \textcolor[rgb]{1.00,0.00,0.00}{小伙伴}\\ \texttt{xiaohuoban@pku.edu.cn}
        }
\date[,,]{\textcolor[rgb]{0.00,1.00,0.00}{\today}}
\frame{\titlepage}
%------------------------------------------------------------------------
\begin{frame}
 \frametitle{\textcolor[rgb]{1.00,0.00,1.00}{索引}}
 \tableofcontents%[hideallsubsections]
\end{frame}
%\frame{\tableofcontents}
%------------------------------------------------------------------------
\section{太空}
\label{LabelNameCannotBeChinese1}
\subsection{宇宙交响乐}
\label{LabelNameCannotBeChinese2}
\begin{frame}{亘古无垠}
 \textsf{心无界!}
\end{frame}
%------------------------------------------------------------------------
\section{梦幻}
\label{LabelNameCannotBeChinese3}
\subsection{梦的幻想曲}
\label{LabelNameCannotBeChinese4}
\begin{frame}%[shrink]%自动压缩
 \frametitle{轻盈绚烂}
 \texttt{梦无疆!}
\end{frame}
%------------------------------------------------------------------------
\section{Beamer效果}
\subsection{逐行显示的实现}
\frame{
  \frametitle{\subsecname}
  \begin{itemize}[<+-|alert@+>]
%beamer取下面两个数字中最大的一个作为一个frame所显示Slide的帧数:
%1,<>里的最大数值,其中<>里的数值表明项目或效果在那些帧上显示
%2,未加<>的item的项数,这需要有[<+->]或[<-+>]参数,并需要有列举环境
%注意,[<-+>]是逐项消失
%[<+->]或[<-+>]的位置决定这两个参数的作用范围
   \item<alert@-1,3-> 在第1,3张上被加亮
   \item 从第1张开始出现,并在第1张上被加亮
   \item 从第2张开始出现,并在第2张上被加亮
   \item 从第3张开始出现,并在第3张上被加亮
  \end{itemize}
}
%------------------------------------------------------------------------
\subsection{字体和色彩的演示}
\frame{
  \frametitle{\subsecname}
  1.\alt<1>{{\kai 楷体只在第1张上}}{楷体只在第1张上}\\
  2.{\color<1,2-3>[rgb]{0.00,1.00,1.00}在第1,2,3张上是蓝的}\\
  3.{\color<3>[rgb]{1.00,0.00,0.00}在第3张上是红的}\\
  4.\alert<-3>{alert代表红色}\\
  5.\structure<5->{structure代表绿色}\\
  \only<6>{6.仅在第6张出现}
}
%------------------------------------------------------------------------
\subsection{换页动态效果}
\frame[<+->]{
  \frametitle{\subsecname}
  \begin{enumerate}
  \item 水平出现效果
    \transblindshorizontal<1>
  \item 竖直出现效果
    \transblindsvertical<2>
  \item 从中心到四角
    \transboxin<3>
  \item 从四角到中心
    \transboxout<4>
  \item 溶解效果
    \transdissolve<5>
  \item Glitter
    \transglitter<6>
  \item 竖直撕开(向内)
    \transsplitverticalin<7>
  \item 竖直撕开(向外)
    \transsplitverticalout<8>
  \item 涂抹
    \transwipe<9>
  \item 渐出
    \transduration<10>{1}
  \end{enumerate}
}
%------------------------------------------------------------------------
\subsection{设置待现内容的样式}
\setbeamercovered{highly dynamic}%[<+->]以及uncover的显示效果由setbeamercovered命令来设定
\begin{frame}
 \begin{block}{待现内容是半透明还是不可见?}
  \uncover<1->{一直不透明}
  \uncover<1,3->{在第2张透明}
  \uncover<1,4->{在第2,3张透明}
 \end{block}
\end{frame}
\setbeamercovered{invisible}
\begin{frame}
 \begin{block}{待现内容是半透明还是不可见?}
  \uncover<1->{一直出现}
  \uncover<1,3->{在第3张重现}
  \uncover<1,4->{在第4张重现}
 \end{block}
\end{frame}
\begin{frame}{Frame Number}
 \text{} {} {} {} {} {} {} FrameNumber :\insertframenumber

 TotalFrameNumber :\inserttotalframenumber
\end{frame}
%------------------------------------------------------------------------
\subsection{图片的使用}
\begin{frame}
 \begin{columns}%多栏形式
  \begin{column}{5cm}
   \pgfdeclareimage[width=5cm]{pkulogo}{/home/sagittarius/LaTeX/figure/pkulogo.png}%声明图片,使用pgfuseimage来使用图片
   \pgfuseimage{pkulogo}<1>
   \pgfdeclareimage[width=5cm]{pkulogo3}{/home/sagittarius/LaTeX/figure/pkulogotrans.png}
   \pgfuseimage{pkulogo3}<2>
  \end{column}
  \begin{column}{2cm}
   \begin{itemize}
    \item <1-|alert@1>$q_1$
    \item <2-|alert@2>$q_2$
   \end{itemize}
  \end{column}
 \end{columns}
\end{frame}
%------------------------------------------------------------------------
\subsection{超级链接的实现}
\frame{
  \frametitle{\subsecname}
  \hypertarget<1>{gotofirst}{欢迎来到第1页\\}
  \hypertarget<5>{gotofifth}{欢迎来到第5页\\}
  \begin{itemize}
   \item<2-> 使用 \textbf{hypertarget} 命令添加链接目标
    \hyperlinkframestartnext{\beamerskipbutton{略过}}
   \item<3-> 使用 \textbf{hyperlink} 命令添加链接跳转
   \item<4->
    \hyperlink{gotofifth}{\beamergotobutton{到第5页}}
   \item<1->
    \beamerbutton{到第\pageref{LabelNameCannotBeChinese1}页上的%
                  第\ref{LabelNameCannotBeChinese1}节第\ref{LabelNameCannotBeChinese2}小节}\\
    \beamerbutton{到第\pageref{LabelNameCannotBeChinese3}页上的%
                  第\ref{LabelNameCannotBeChinese3}节第\ref{LabelNameCannotBeChinese4}小节}\\
   \item<5->
    \hyperlink{zhixie}{\beamerbutton{回家}}
   \item<6->
    \hyperlink{gotofirst}{\beamerreturnbutton{回第1页}}
  \end{itemize}
}
%------------------------------------------------------------------------
\subsection{使用verbatim环境}
\begin{frame}[fragile]
\begin{verbatim}
for \alert(i) in range(10):
 print \alert(i)
\end{verbatim}
\begin{semiverbatim}
for \alert{i} in range(10):
 print \alert{i}

在semiverbatim环境中, \\、 \{、 \} 是有特殊意义的
\end{semiverbatim}
\end{frame}
%------------------------------------------------------------------------
\subsection{插入多媒体}
\begin{frame}{视频}
 \movie[externalviewer]{来看一段视频!}{/home/sagittarius/LaTeX/video/Change1.wmv}
\end{frame}
\begin{frame}{音频}
 \movie[externalviewer]{这里有一段音乐}{/home/sagittarius/LaTeX/audio/Helene.mp3}
\end{frame}
%------------------------------------------------------------------------
\renewcommand{\appendixname}{附录}
\appendix
\section{结束语}
\begin{frame}%[shrink=5]%缩小最多不超过5%
 \frametitle{\secname}
 \begin{center}
  \textbf{\Huge{\textcolor[rgb]{1.00,0.00,0.00}{\kai 漫步宇内}}}
 \end{center}
 \begin{center}
  \textbf{\Huge{\textcolor[rgb]{0.00,0.50,1.00}{\hei 昂首天外}}}
 \end{center}
\end{frame}
\section{致谢}
\begin{frame}[label=zhixie]
 \begin{center}
  \textbf{\Huge{\textcolor[rgb]{1.00,0.00,1.00}{\kai 谢谢大家}}}
 \end{center}
\end{frame}
%------------------------------------------------------------------------
\end{document}
